% Created by Kamil Niemcewicz 

%----------------------------------------------------------------------------------------
%	 left panel
%----------------------------------------------------------------------------------------

% * * * * * * * * *
% * Creates header and prints in new line every text passed in 2nd argument
% * * * * * * * * *
\newcommand{\profilesection}[2]%
{%
    \vspace{8pt}%
    {%
        \color{leftPanelTextHead} \huge #1 \\\rule[0.15\baselineskip]{6.09cm}{1pt}%
    }%
    
    \foreach \n in #2%
    {%
        \lineOfText \n%
    }%
}

% * * * * * * * * *
% * Prints colored text
% * * * * * * * * *
\newcommand{\lineOfText}[1]{\color{leftPanelText} #1\\}


% * * * * * * * * *
% * Prints colored text with progress bar from valued 0.1 to 5.0
% * * * * * * * * *
\newcommand{\lineOfTextWPB}[2]{%
#1\\%
	\begin{tikzpicture}
		\draw[fill=white,leftPanelRectangle](0,0.8) rectangle (#2 ,0.4);
		\node [above right] at (0,0.4) {};
	\end{tikzpicture}
\\%
	}

%----------------------------------------------------------------------------------------
%	 right panel
%----------------------------------------------------------------------------------------

% * * * * * * * * *
% * Just renewing the section command. 
% * First argument takes header text and coloring it's first three letters.
% * * * * * * * * *
\renewcommand{\section}[1]{
	{%
		\headingfont\color{headercolor}\LARGE\firstThreeColored #1\par%
	}
}

\newenvironment{template}
{%
	\begin{tabular*}{\textwidth}
	{
	    @{\extracolsep{\fill}}
    ll}
   
}
    
% * * * * * * * * *
% * Section item.
% * Arguments:
% * 1st - date
% * 2nd - company name 
% * 3rd - text indented to the right side on 2nd argument's line
% * 4rd - description
% * * * * * * * * *
\newcommand{\templateitem}[4]{%
	#1&\parbox[t]{0.84\textwidth}{%
		\textbf{#2}% 
		\hfill%
		{\footnotesize#3}\\%}
		#4\vspace{\parsep}%
	}\\
}

% * * * * * * * * *
% * Section item without indent
% * Arguments:
% * 1st - date
% * 2nd - company name 
% * 3rd - text indented to the right side on 2nd argument's line
% * 4rd - description
% * * * * * * * * *
\newcommand{\templateZeroIndent}[3]{%
	\parbox[t]{1.03\textwidth}{%
		\textbf{#1}% 
		\hfill%
		{\color{red}#2}\\%}
		#3\vspace{\parsep}%
	}\\
}

% * * * * * * * * *
% * Takes word, color first three letters and returns it.
% * * * * * * * * *
\def\firstThreeColored#1#2#3{%
	{%
		\color{red} #1#2#3%
	}%
}
